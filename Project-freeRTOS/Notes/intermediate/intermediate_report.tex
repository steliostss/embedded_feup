\documentclass{article}
\usepackage[utf8]{inputenc}

\title{Embedded - Intermediate Report}
% \subtitle{Assignment 10: Install and characterize an RTOS}
\author{Vasileios Konstantaras, Alena Tesarova, Stylianos Tsagkarakis}
\date{25 April 2020}

\usepackage{natbib}
\usepackage{graphicx}
\usepackage{hyperref}
\hypersetup{
    colorlinks=true,
    linkcolor=blue,
    filecolor=magenta,      
    urlcolor= blue,
}

\urlstyle{same}
\begin{document}

\maketitle

\section{RTOS Selection}
We have selected FreeRTOS as the RTOS we are going to install and evaluate for this assignment. Since there is a lack of hardware we decided to go with this because there is a simulation of FreeRTOS through a Windows port (for more info see \href{https://freertos.org/FreeRTOS-Windows-Simulator-Emulator-for-Visual-Studio-and-Eclipse-MingW.html}{here}), thus it will be easier for us. 

\section{Metrics for Benchmark}
The metrics we are going to evaluate separate in three sets: 

\begin{enumerate}

    \item Latency
        \begin{itemize}
            \item System calls latency
            \item Latency of peripheral functions
            \item Context switch overhead
            \item Jitter causes by OS overhead
        \end{itemize}

    \item Performance / Throughput
        \begin{itemize}
            \item Memory usage
            \item Program memory usage
            \item Reliability and determinism
        \end{itemize}

    \item OS Features
        \begin{itemize}
            \item Supported algorithms
            \item Real-time tracing features
            \item Support of static memory allocation
            \item Security
            \item OS Support / Community support
        \end{itemize}

\end{enumerate}

% \newpage
\clearpage

\section{Implementation}

While benchmarking FreeRTOS we should follow the requirements mentioned below:
\begin{itemize}
    \item The measurements should be as accurate as possible.
    \item The solutions developed should be as non-intrusive as possible.
    \item The measurement should give average results.
\end{itemize}

\noindent \textbf{Example and principles} \footnote[1]{All of the above will be tested in custom scripts written by us to test the system both normal execution load and under extreme load.}

\begin{itemize}

\item The latency of functions can be examined by measuring the execution time. The on-
chip timer based software tool gives timestamps at both the beginning and the end of
a given function. It is suggested to use two methods of measuring time.

\item The memory usage of a given data structure can be measured by calculating the amount of heap used in the \textit{malloc()} function when the concerned structure is created.

\item The program code size is shown by a map file created by the compiler.

\end{itemize}

% \end{section}

% \bibliographystyle{plain}
% \bibliography{references}
\end{document}

